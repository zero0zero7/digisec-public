\documentclass[10pt,a4paper]{article}
\usepackage{geometry}
\usepackage{graphicx}
\usepackage{helvet}

\usepackage{epstopdf}
\usepackage{tikz}
\usetikzlibrary{calc, chains, positioning, shadings, shadows, shapes.arrows}

\usepackage{ctable}

\renewcommand{\familydefault}{\sfdefault}

% Styles for interfaces and edge labels
\tikzset{%
  interface/.style={draw, rectangle, rounded corners, font=\LARGE\sffamily},
  ethernet/.style={interface, fill=yellow!50},% ethernet interface
  serial/.style={interface, fill=green!70},% serial interface
  speed/.style={sloped, anchor=south, font=\large\sffamily},% line speed at edge
  route/.style={draw, shape=single arrow, single arrow head extend=4mm,
    minimum height=1.7cm, minimum width=3mm, white, fill=blue!20,
    drop shadow={opacity=.8, fill=blue!50!black}, font=\tiny}% inroute / outroute arrows
}
\newcommand*{\shift}{1.3cm}% For placing the arrows later


\begin{document}

\title{Proposed details for SIS PILOT}
\maketitle{}

The proposed external-facing details for the SIS Pilot.

Note: this document should be read in conjunction with \emph{Proposed details for SIS deployment}.

\section{SIS Pilot on DSNET}

\subsection{Network logical architecture}

\begin{figure}[h]
\centering
\scalebox{0.8}{
\begin{tikzpicture}[node distance=4cm]
    \node (DSN) [
        label=left:DSNET clients
    ] {\includegraphics{3015_eps/terminal}};

    \node (DSNFW) [
        right of=DSN
    ] {\includegraphics{3015_eps/firewall}};

    \node [
        align=center,
        anchor=north
    ] at (DSNFW.south) {Permit from DSN\\Deny to DSN\\SIS HTTPS only};

    \node (DSN_SW) [
        right of=DSNFW,
        label=right:DSNET TOR
    ] {\includegraphics{3015_eps/workgroup_switch}};

    \node (SIS_GW1) [
        below of=DSNFW,
        label=right:SIS-Pilot-1
    ] {\includegraphics{3015_eps/standard_host}};

    \node (SIS_INT1) [
        below of=SIS_GW1,
        yshift=15mm,
        label=right:TOR
    ] {\includegraphics{3015_eps/workgroup_switch}};

    \node (SIS_DMZ_a) [coordinate, at=(SIS_GW1), xshift=-1cm, yshift=1cm] {};
    \node (SIS_DMZ_b) [coordinate, at=(SIS_GW1), xshift=-1cm, yshift=-1cm] {};
    \node (SIS_DMZ_c) [coordinate, at=(SIS_GW1), xshift=8cm, yshift=-1cm] {};
    \node (SIS_DMZ_d) [coordinate, at=(SIS_GW1), xshift=8cm, yshift=1cm, label=below left:SIS DMZ (RT)] {};
    \draw [-,thick,gray] (SIS_DMZ_a) -- (SIS_DMZ_b) -- (SIS_DMZ_c) -- (SIS_DMZ_d) -- (SIS_DMZ_a);

    \node (DSN_ADM_FW) [
        below of=DSN
    ] {\includegraphics{3015_eps/firewall}};

    \node [
        align=center,
        anchor=east
    ] at (DSN_ADM_FW.west) {Permit SIS administrative\\clients only};

    \draw [-, thick, gray]
    (DSN) -- (DSN_ADM_FW) -- (SIS_GW1);

    \node (SIS_INT) [
        below=of SIS_INT1,
        yshift=20mm,
        label=center:Menlo
    ] {\includegraphics{3015_eps/cloud}};

    \draw [-, thick, black]
    (DSN) -- (DSNFW) -- (DSN_SW) -- (SIS_GW1) -- (SIS_INT1) -- (SIS_INT);

\end{tikzpicture}
}
\caption{SIS logical architecture on DSNET.}
\end{figure}


\subsection{Host overview}

\subsubsection{SIS-Pilot-1}

This is the DSNET client-facing pilot machine.

\subsection{IP assignments}

\begin{table}[h!]
    \caption{Requested IP assignments}
    \begin{tabular}{l l l}
        Host & Assignment & Externally routable? \\
        \hline
        SIS-Pilot-1 & IP1 (to be assigned) & Y \\
    \end{tabular}
\end{table}


\subsection{DNS records}

\begin{table}[h!]
    \caption{Proposed DNS names}
    \begin{tabular}{l l l}
        Hostname & Record type & Data \\
        \hline
        pilot.sis.dsnet.dso.root. & A & \emph{IP1} \\
    \end{tabular}
\end{table}

A zonefile will be included here when the names and data are confirmed.

\subsection{Firewall access}

\ctable[
    caption=Routes to be permitted,
    pos=h!
]{p{50mm} p{95mm}}{}{
    \FL
    Source & Dest \ML
    DSNET end-user clients \newline (any high-numbered port) & IP1 \newline (80/tcp, 443/tcp) \ML
    Designated administrative clients \newline (any high-numbered port) & IP1 \newline (22/tcp) \ML
    SIS-Pilot-1 \newline (any high-numbered port) & 13.229.252.0/25 \newline (443/tcp, 3129/tcp, 9443/tcp) \ML
    SIS-Pilot-1 \newline (any high-numbered port) & pac.menlosecurity.com \newline (443/tcp) \ML
    SIS-Pilot-1 \newline (any high-numbered port) & *proxy0-418f1090cac5bb897d919b628e87950c.menlosecurity.com \newline (443/tcp, 3129/tcp, 9443/tcp) \ML
    SIS-Pilot-1 \newline (any high-numbered port) & *proxy1-418f1090cac5bb897d919b628e87950c.menlosecurity.com \newline (443/tcp, 3129/tcp, 9443/tcp) \ML
    SIS-Pilot-1 \newline (any high-numbered port) & DSO external-facing DNS server \newline (53/tcp, 53/udp) \LL
}

\subsection{TLS certificates}

TLS certificates with names specified in Table~\ref{tab:dsnet:tls} are required to deploy the pilot service.
These certificates \emph{must} be rooted in a CA trusted by SIS Pilot clients.

\ctable[
    caption=TLS certificates required,
    label=tab:dsnet:tls,
    pos=h!
]{l l}{}{
    \FL
    Subject name & Details \ML
    pilot.sis.dsnet.dso.root & SAN: pilot.sis (to check) \LL
}


\section{SIS Pilot on TN}

\subsection{Network logical architecture}

\begin{figure}[h]
\centering
\scalebox{0.8}{
\begin{tikzpicture}[node distance=4cm]
    \node (DSN) [
        label=left:TN clients
    ] {\includegraphics{3015_eps/terminal}};

    \node (DSNFW) [
        right of=DSN
    ] {\includegraphics{3015_eps/firewall}};

    \node [
        align=center,
        anchor=north
    ] at (DSNFW.south) {SIS HTTPS only};

    \node (DSN_SW) [
        right of=DSNFW,
        label=right:TN TOR
    ] {\includegraphics{3015_eps/workgroup_switch}};

    \node (SIS_GW1) [
        below of=DSNFW,
        label=right:SIS-Pilot-2
    ] {\includegraphics{3015_eps/standard_host}};

    \node (SIS_INT1) [
        below of=SIS_GW1,
        yshift=15mm,
        label=right:TOR
    ] {\includegraphics{3015_eps/workgroup_switch}};

    \node (SIS_DMZ_a) [coordinate, at=(SIS_GW1), xshift=-1cm, yshift=1cm] {};
    \node (SIS_DMZ_b) [coordinate, at=(SIS_GW1), xshift=-1cm, yshift=-1cm] {};
    \node (SIS_DMZ_c) [coordinate, at=(SIS_GW1), xshift=8cm, yshift=-1cm] {};
    \node (SIS_DMZ_d) [coordinate, at=(SIS_GW1), xshift=8cm, yshift=1cm, label=below left:SIS DMZ (RT)] {};
    \draw [-,thick,gray] (SIS_DMZ_a) -- (SIS_DMZ_b) -- (SIS_DMZ_c) -- (SIS_DMZ_d) -- (SIS_DMZ_a);

    \node (DSN_ADM_FW) [
        below of=DSN
    ] {\includegraphics{3015_eps/firewall}};

    \node [
        align=center,
        anchor=east
    ] at (DSN_ADM_FW.west) {Permit SIS administrative\\clients only};

    \draw [-, thick, gray]
    (DSN) -- (DSN_ADM_FW) -- (SIS_GW1);

    \node (SIS_Menlo) [
        below=of SIS_INT1,
        yshift=20mm,
        label=center:Menlo
    ] {\includegraphics{3015_eps/cloud}};

    \node (SIS_INT) [
        right=of SIS_Menlo,
        label=center:Internet
    ] {\includegraphics{3015_eps/cloud}};

    \draw [-, thick, black]
    (DSN) -- (DSNFW) -- (DSN_SW) -- (SIS_GW1) -- (SIS_INT1) -- (SIS_INT)
    (SIS_INT1) -- (SIS_Menlo);

\end{tikzpicture}
}
\caption{SIS logical architecture on Technet.}
\end{figure}


\subsection{Host overview}

\subsubsection{SIS-Pilot-2}

This is the TN client-facing pilot machine.
It will be used for testing and building.

\subsection{IP assignments}

\begin{table}[h!]
    \caption{Requested IP assignments}
    \begin{tabular}{l l l}
        Host & Assignment & Externally routable? \\
        \hline
        SIS-Pilot-2 & IP1 (to be assigned) & Y \\
    \end{tabular}
\end{table}


\subsection{DNS records}

The DNS zone is to be confirmed.

\begin{table}[h!]
    \caption{Proposed DNS names}
    \begin{tabular}{l l l}
        Hostname & Record type & Data \\
        \hline
        test.sis & A & \emph{IP1} \\
    \end{tabular}
\end{table}

A zonefile will be included here when the names and data are confirmed.

\subsection{Firewall access}

\ctable[
    caption=Routes to be permitted,
    pos=h!
]{p{50mm} p{95mm}}{}{
    \FL
    Source & Dest \ML
    TN clients \newline (any high-numbered port) & IP1 \newline (80/tcp, 443/tcp) \ML
    Designated administrative clients \newline (any high-numbered port) & IP1 \newline (22/tcp) \ML
    SIS-Pilot-2 (any port) & Internet (any port) \LL
}

\subsection{TLS certificates}

TLS certificates with names specified in Table~\ref{tab:dsnet:tls} are required to deploy the pilot service.
These certificates \emph{must} be rooted in a CA trusted by TN clients.

Note: the Subject name should be modified when the DNS zone is known!

\ctable[
    caption=TLS certificates required,
    label=tab:dsnet:tls,
    pos=h!
]{l l}{}{
    \FL
    Subject name & Details \ML
    test.sis &  \LL
}

\end{document}